\documentclass[a4paper,10pt]{ltjsarticle}
\usepackage{luatexja}
\RequirePackage{plautopatch}

% --- 主要パッケージ ---
\usepackage{graphicx}     % 画像
\usepackage{amsmath}      % 数式
\usepackage{color}        % 文字色
\usepackage{url}          % URL
\usepackage{here}         % 図表の位置固定 [H]
\usepackage{titlesec}     % セクション見出しの書式設定

% --- biblatexによる参考文献設定 (biberバックエンド) ---
\usepackage[
    backend=biber,
    style=authoryear, % 引用スタイル(authoryear)
    sorting=nyt,       % 引用順序(名前、年、タイトル)
    natbib=true         % natbib互換コマンドを有効化
]{biblatex}
\addbibresource{reference.bib} % .bibファイルを指定

% --- ハイパーリンク設定 ---
\usepackage[luatex,pdfencoding=auto]{hyperref}
\hypersetup{
    setpagesize=false,
    bookmarksnumbered=true,
    bookmarksopen=true,
    colorlinks=true,
    linkcolor=red,   % 内部リンクの色
    citecolor=black, % 参考文献へのリンクの色
    urlcolor=blue    % URLの色
}

% --- 文書全体の書式設定 ---
\titleformat*{\section}{\mcfamily\Large}
\titleformat*{\subsection}{\mcfamily\large}
\titleformat*{\subsubsection}{\mcfamily\normalsize}
\pagenumbering{arabic} % ページ番号をアラビア数字に設定

% --- 文書情報 ---
\title{プログラミング演習 最終課題レポート}
\author{08-252019 小倉直己}
\date{\empty}

\begin{document}

\maketitle

% --- 本文 ---
\section{選択した課題}
今回は、潜水艦ゲームの課題を選択した。

\section{AIのアルゴリズムの説明}
今回作成したAIは、以下のアルゴリズムに則って動作するようにした。

\subsection{船の配置}

味方の艦の配置は、全ての艦が以下のルールを満たすように配置される。
\begin{itemize}
    \item 艦は$5 \times 5$のマス目の中に配置される。
    \item ある艦から半径2マスの正方形の範囲内に他の艦が存在しない。
\end{itemize}

この配置を実装すると、以下のプログラムのようになる。

\begin{verbatim}
def place_ship(self):
        distance = 2  # 2マス以上離す
        placed_positions = set()
        ship_types = ['w', 'c', 's']
        max_attempts = 500
        for ship_type in ship_types:
            placed = False
            attempts = 0
            while not placed and attempts < max_attempts:
                attempts += 1

                # choice a random position which is not occupied, and scattered.
                row = self.rng.randint(0, self.field.height - 1)
                col = self.rng.randint(0, self.field.width - 1)
                position = [col, row]

                if tuple(position) not in placed_positions:
                    is_isolated = True
                    for placed_pos in placed_positions:
                        if abs(position[1] - placed_pos[1]) <= distance and abs(position[0] - placed_pos[0]) <= distance:
                            # If the new position is too close to any placed ship, break
                            is_isolated = False
                            break
                    
                    if is_isolated:
                        placed_positions.add(tuple(position))
                        placed = True
                        logging.info(f"Placed {ship_type} at {position}")
            if not placed:
                logging.warning(f"Failed to place {ship_type} after {max_attempts} attempts. Trying to place in a random way.")
                for r in range(self.field.height):
                    for c in range(self.field.width):
                        if (c, r) not in placed_positions:
                            placed_positions.add((c, r))
                            logging.info(f"Placed {ship_type} at {[c, r]}")
                            break
                    if placed:
                        break

        return {
            'w': list(list(placed_positions)[0]),
            'c': list(list(placed_positions)[1]),
            's': list(list(placed_positions)[2])
        }
\end{verbatim}

このコードは、今回作成したoriginal\_player.pyの一部であり、OriginalPlayerクラスのplace\_shipメソッドとして実装されている。
このプログラムでは、ランダムに艦を配置する試行を100回繰り返し、先ほど述べたルールに従って配置できなかった場合は、ルールと関係なくランダムに配置するようにしている。



\printbibliography

\end{document}